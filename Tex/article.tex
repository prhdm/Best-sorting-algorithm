\documentclass[
10pt, % Main document font size
a4paper, % Paper type, use 'letterpaper' for US Letter paper
oneside, % One page layout (no page indentation)
%twoside, % Two page layout (page indentation for binding and different headers)
headinclude,footinclude, % Extra spacing for the header and footer
BCOR5mm, % Binding correction
]{scrartcl}
\usepackage{hyperref}
\usepackage{graphicx} 
\usepackage{lettrine}
\usepackage{listings}
\usepackage{color}
\definecolor{dkgreen}{rgb}{0,0.6,0}
\definecolor{gray}{rgb}{0.5,0.5,0.5}
\definecolor{mauve}{rgb}{0.58,0,0.82}
\lstset{frame=tb,
  language=Java,
  aboveskip=3mm,
  belowskip=3mm,
  showstringspaces=false,
  columns=flexible,
  basicstyle={\small\ttfamily},
  numbers=none,
  numberstyle=\tiny\color{gray},
  keywordstyle=\color{blue},
  commentstyle=\color{dkgreen},
  stringstyle=\color{mauve},
  breaklines=true,
  breakatwhitespace=true,
  tabsize=3
}
\input{structure.tex} 

\hyphenation{Fortran hy-phen-ation} 
\title{\includegraphics[scale=0.2]{Sharif-Logo.png}\\\normalfont\spacedallcaps{DETERMINING THE PROPER SORTING ALGORITHM BASED ON THE GIVEN DATA}}

\author{\spacedlowsmallcaps{Parham Houshmand, Yasamin Vaziri 
} \\
\spacedlowsmallcaps{Mahya Mottaghi \& Elham Soleimani}
}

\date{
\spacedlowsmallcaps{Sharif University of Technology}
\\
July 2022 }

\begin{document}


\renewcommand{\sectionmark}[1]{\markright{\spacedlowsmallcaps{#1}}} % The header for all pages (oneside) or for even pages (twoside)
%\renewcommand{\subsectionmark}[1]{\markright{\thesubsection~#1}} % Uncomment when using the twoside option - this modifies the header on odd pages
\lehead{\mbox{\llap{\small\thepage\kern1em\color{halfgray} \vline}\color{halfgray}\hspace{0.5em}\rightmark\hfil}} % The header style

\pagestyle{scrheadings} % Enable the headers specified in this block

%----------------------------------------------------------------------------------------
%	TABLE OF CONTENTS & LISTS OF FIGURES AND TABLES
%----------------------------------------------------------------------------------------

\maketitle % Print the title/author/date block

\setcounter{tocdepth}{2} % Set the depth of the table of contents to show sections and subsections only

\tableofcontents % Print the table of contents
\newpage
%\listoffigures % Print the list of figures

%\listoftables % Print the list of tables

%----------------------------------------------------------------------------------------
%	ABSTRACT
%----------------------------------------------------------------------------------------

\section{Abstract}
\lettrine[findent=2pt]{\textbf{T}}{ }he cost of using sorting algorithms may have a considerable difference in that finding the proper algorithm reduces time and space and improves sorting efficiency.
There are linear sorting algorithms that need to be preprocessed before they can be used.
The quick sort algorithm also needs the data to be unsorted and sometimes according to the structured data we have (heap, linked list) they need another special algorithm (heap and merge sort) also sometimes the data is very voluminous, and We have to use an algorithm that performs in place sorting. According to these interpretations, we perform linear pre-processing on the primary data to find the proper sorting algorithm with the features we mention from each sorting algorithm.
\begin{center}
 \emph{keywords}:algorithm, sorting, data, input, proper sorting algorithm, time complexity
\end{center}
 
\section{Introduction}

As there are different sorting algorithms, we know they have  
bold differences with each other and we are going to give a short summary about them. But the main reason of this paper is to construct the proper sorting algorithm with given input which has the minimum time complexity no matter what the size of input is.Our algorithm firstly checks each number behind the head whether it is ascending or descending, also checks whether the number is negative or not, then it finds the largest length of the number and the largest number, after that it checks its inversion and then it finds whether the inversion is larger than n$\log$ n or not. Note that for radix and count sort the ascending order should not neither be less than n$\log n$ nor negative.

\section{Methods}



\begin{enumerate}[noitemsep]
\item different methods of sorting
\item different techniques
\item Analysis of sorting techniques 
\item proper sorting algorithm analysis
\end{enumerate}

%------------------------------------------------

\subsection{different sorts time complexity}

\paragraph{}  
\begin{enumerate}[noitemsep]
\item  Bubble sort and Insertion sort
\\
The average and worst case time complexity of both is $n^2 $ while the best time complexity is n. This happens when the array is already sorted, instead the worst case is when the array is completely reversed. 
and as a use, these algorithms work better for the smaller data and insertion sort is of order of n when the array is sorted.


\item Merge sort
\\
Merge sorts best, average and worst case time complexity is$nlogn$ and again similarly independent of distribution of data. for this algorithm we can say it works the best when the array is reversed sort

\item Heap sort
\\
like merge sort, its best, average and worst case time complexity is $nlogn$ which is independent of distribution of data. 
heap sort's property is that the space is of order of 1.
\item Quick sort
\\
It is a divide and conquer approach with recurrence relation: 
$$T(n) = T(k) + T(n-k-1) + cn$$
Its worst case is when the array is sorted or reverse sorted, the partition algorithm divides the array in two sub arrays with $0$ and $n-1$ elements. Therefore, 
$$T(n) = T(0) + T(n-1) + cn$$
Solving this we get:
$$T(n) = O(n^2)$$
On an average, the partition algorithm divides the array in two subarrays with
 equal size. Therefore,
$$T(n) = 2T(\frac{n}{2}) + cn$$
and finally
$$T(n) = O(nlogn)$$
the algorithm is of order of $n^2$ when the array is revered.
Non-comparison based sorting:
\\
In non-comparison based sorting, elements of array are not compared with each other to find the sorted array.
\\

\item Radix sort
\\
Best, average and worst case time complexity for radix sort is $nk$ where $k$ is the maximum number of digits in elements of array. 
\\

\item Count sort
\\
For this algorithm best, average and worst case time complexity is $n+k$ where k is the size of count array.



\end{enumerate}




%------------------------------------------------

\subsection{common techniques}
\begin{enumerate}[noitemsep]
\item in place \& out place technique
\\
A sorting technique is in place if it does not use any extra memory to sort the array. 
Among the comparison based techniques discussed, only merge sort is outplaced technique as it requires an extra array to merge the sorted sub arrays. 
Among the non-comparison based techniques discussed, all are outplaced techniques. Counting sort uses a counting array and bucket sort uses a hash table for sorting the array. 

\item on line \& off line technique
\\
A sorting technique is considered Online if it can accept new data while the procedure is ongoing i.e. complete data is not required to start the sorting operation. 
Among the comparison based techniques discussed, only Insertion Sort qualifies for this because of the underlying algorithm it uses i.e. it processes the array (not just elements) from left to right and if new elements are added to the right, it doesn’t impact the ongoing operation. 

\item Stable \& unstable technique
\\
A sorting technique is stable if it does not change the order of elements with the same value. 
Out of comparison based techniques, bubble sort, insertion sort and merge sort are stable techniques. Selection sort is unstable as it may change the order of elements with the same value. For example, consider the array $4, 4, 1, 3.$ 
In the first iteration, the minimum element found is 1 and it is swapped with 4 at 0th position. Therefore, the order of 4 with respect to 4 at the 1st position will change. Similarly, quick sort and heap sort are also unstable. 
Out of non-comparison based techniques, Counting sort and Bucket sort are stable sorting techniques whereas radix sort stability depends on the underlying algorithm used for sorting. 
\end{enumerate}



\subsection{sorting techniques analysis}

\begin{enumerate}[noitemsep]
\item When the array is almost sorted, insertion sort can be preferred.
\item When order of input is not known, merge sort is preferred as it has worst case time complexity of $nlogn$ and it is stable as well.
\item When the array is sorted, insertion and bubble sort gives complexity of n but quick sort gives complexity of $n^2$.

\end{enumerate}






\subsection{Constructed Algorithm}
 firstly it checks the inversion number of the array then it checks that if it is ascending or descending and if it has negative element and it finds max value and max value digits then it checks if number of inversions are less  than nlogn and number of ascending and descending numbers are equal to the length of the array or the length of the array is (0,20] the we sort with insertion. 
 if it is only descending (reversed sorted)we sort with merge sort and if it is completely ascending we use insertion.
 now if it doesn't contain any negative number we check if the max value is less than nlogn we use count sort and if max number of digits is less than nlogn we return radix sort.if none of these algorithms were not the answer we choose between quick and merge sort. if number of ascending and descending numbers are less than n we use quick sort, else we return merge sort.
 \\
The python code of the algorithm would be like:

\begin{lstlisting}
from math import log


def pyme(array: list):
    """
    This function is the implementation of the algorithm.
    It takes an array as input and returns the best sorting algorithm.
    """
    """
    The complexity of this algorithm is O(n log n).
    we can improve the algorithm by using the following optimizations:
    1. We can ignore the inversions because we use it for determining the best sorting algorithm on small arrays
    and we can ignore it for large arrays.
    2. because we use the max_value and max_number_of_digits to determine the best sorting algorithm for arrays with
    no negative numbers, we can ignore them for arrays with negative numbers and we can just find 
    the max number of digits by the number that has the max value.
    """
    is_asc, is_desc, has_negative, max_value, max_number_of_digits = __analyse(array)  # Analyse the array for
    # finding attributes of the array
    inversions = inversion_count(array)  # Count the inversions for Insertion Sort
    n_log_n = len(array) * log(len(array), 2)  # Calculate the n log n
    if inversions < n_log_n or (is_asc == len(array) - 1 and is_desc == len(array) - 1) or \
            20 >= len(array) > 1 or is_asc == len(array) - 1:
        return "Insertion Sort"  # If the inversions are less than n log n or
        # the array is sorted in ascending order or if the array is small or all keys are the same we use Insertion Sort
    if is_desc == len(array) - 1:
        return "Merge Sort"  # If the array is sorted in descending order we use Merge Sort
        # by test, it works better than others
    if not has_negative:  # count sort and radix sort can only be used if the array has no negative numbers
        if max_number_of_digits < log(len(array), 2) and max_number_of_digits * len(array) < max_value + len(array) \
                and max_number_of_digits * len(array) < n_log_n / 2:
            return "Radix Sort"  # for the array with bigger rang than linear and the number of digits is smaller
            # than the biggest number of the array we use Radix Sort
        if max_value < len(array) * log(len(array), 2):
            return "Count Sort"  # if the max value is small count sort is better than radix sort
    if is_asc < len(array) * 3 / 5 and is_desc < len(array) * 3 / 5:
        return "Quick Sort"  # Quick Sort is better than Merge Sort for arrays with a lot of inversions
        # and if the array is sorted in ascending order and descending order we use other sorts.
        # it works better on big arrays.
    else:
        return "Merge Sort"
        # merge sort needs more space than quick sort and merge is better for small arrays.
        # merge sort need to copy array every time and quick sort need to copy array only once.
    # we didn't use heap sort because it is slower than merge sort and quick sort
    # bubble sort is just a simple sorting algorithm that doesn't use any optimizations.
    # bubble sort is easier to understand and it is just handy to use but by test it works slower than Insertion.


def __analyse(array):
    is_asc, is_desc, has_negative, max_value, max_number_of_digits = 0, 0, False, 0, 0
    temp_asc = 0
    temp_desc = 0
    for i in range(len(array) - 1):
        if array[i] < 0:
            has_negative = True
        if array[i] >= array[i + 1]:
            temp_asc = max(temp_asc, is_asc)
            is_desc = temp_desc + 1
        else:
            temp_asc = temp_asc + 1
            is_desc = max(temp_desc, is_desc)
        max_value = max(array[i], max_value)
        max_number_of_digits = max(len(str(array[i])), max_number_of_digits)
    is_asc = max(is_asc, temp_asc)
    is_desc = max(is_desc, temp_desc)
    max_value = max(array[len(array) - 1], max_value)
    max_number_of_digits = max(len(str(array[len(array) - 1])), max_number_of_digits)
    return is_asc, is_desc, has_negative, max_value, max_number_of_digits


def inversion_count(array):
    """
    This function is the implementation of the inversion algorithm.
    It takes an array as input and returns the number of inversions.
    """
    if len(array) <= 1:
        return 0
    mid = len(array) // 2
    left = array[:mid]
    right = array[mid:]
    return inversion_count(left) + inversion_count(right) + merge(left, right)


def merge(left, right):
    count = 0
    i = 0
    j = 0
    while i < len(left) and j < len(right):
        if left[i] <= right[j]:
            i += 1
        else:
            count += len(left) - i
            j += 1
    return count

\end{lstlisting}
the order of the algorithm is: O(n $\log$ n)
\\
for more information check the \href{https://github.com/prhdm/Best-sorting-algorithm}{Git hub} link.
\\

\begin{figure}[h]
\centerline{%
\includegraphics[width=0.5\textwidth]{SMALLNUMMS.png}%
\includegraphics[width=0.5\textwidth]{SORTEDNUMS.png}%
}%
\centerline{%
\includegraphics[width=0.5\textwidth] {NEGATIVE NUMS.png}%
\includegraphics[width=0.5\textwidth] {REVESREDNUMS.png}%
}%
\centerline{%
\includegraphics[width=0.5\textwidth] {DUPLICATENUMS.png}%
}%
\caption{\textbf{Results of testing in diffrenet conditions}}
\label{fig}
\end{figure}


\newpage

\begin{center}

\fbox{\parbox{15cm}{To explain the error of the algorithm, due to the graphs above, this might be because of the CPU, the input, data streaming type, amount of RAM conflict etc. effect the coefficients of the order which are not shown in the order, so that the efficiency will get worse (see figures).
\\
Also we can say that our algorithm's correctness average is between 70 to 80 \% that we discuss later }}
\end{center}




\section{Results and Discussion}


we theoretically made a method to understand the proper sorting  algorithm and that was almost efficient.\\
the algorithm checks different factors in the given data and will return the answer. we also use a test generator for making different tests and try it on our method and compare it with different sorting algorithms to see how well does it work.As an average, it's efficiency is between 70 to 80 \%.
\\
To see complete tests visit \href{https://github.com/prhdm/Best-sorting-algorithm}{Git hub}
\\
see the table below:
\\

\begin{table}[ht]
\centering
\begin{tabular}{|c|c|}
\hline
 factor & efficiency\\
 \hline
 \hline
 sorted numbers & 73 \% \\
 \hline
 small numbers & 81 \% \\
 \hline
 reversed numbers & 87 \% \\
 \hline
 negative numbers & 75 \% \\
 \hline
 large numbers & 70 \% \\
 \hline
 duplicate numbers & 81 \% \\
 \hline 
 duplicate negative numbers & 78 \% \\
 \hline
 

\end{tabular}
\end{table}




The second column is the average of doing 100 tests on the algorithm And as you can see the least efficiency is for large numbers and the most is for duplicate numbers. For more studies it's good to make the method more efficient with a linear order that works the best for every kind of input.


\section{References}
 \href{https://www.geeksforgeeks.org/}{Geeksforgeeks.org}
 \\
 \href{https://www.amazon.com/Introduction-Algorithms-3rd-MIT-Press/dp/0262033844}{Introduction to Algorithms clrs}
 

\paragraph{Team contribution} :
\\
Parham Houshmand:25 \%
\\
Yasamin Vaziri:25 \%
\\
Mahya Mottaghi:25 \% 
\\
Elham Soleimani:25 \%
\\
all participated in discussing problems,paper writing and coding.
\label{fig:esempio}


%----------------------------------------------------------------------------------------
%	BIBLIOGRAPHY
%----------------------------------------------------------------------------------------

\renewcommand{\refname}{\spacedlowsmallcaps{References}} % For 

\bibliographystyle{unsrt}

\bibliography{sample.bib} 


\end{document}
